\documentclass[10pt, a4paper, oneside]{scrartcl}

\usepackage[T1]{fontenc}
\usepackage[utf8]{inputenc}
\usepackage[ngerman]{babel}

\usepackage{amsmath,marvosym}
\usepackage{amssymb}
\usepackage{graphicx}
\usepackage{float}
\usepackage{color}
\usepackage{xcolor}
\usepackage{url}
\usepackage{pdflscape}
\usepackage{multirow}
\usepackage{multicol}
\usepackage{tabularx}
\usepackage{longtable}
\usepackage{array}
\usepackage{listings}

\definecolor{dunkelgrau}{rgb}{0.8,0.8,0.8}
\definecolor{grau}{gray}{0.25}

\newcommand{\header}[5]{
	\noindent\rule{\textwidth}{0.7pt}
		{\centering{\footnotesize 0101010 \ 1110000 \ 1101111 \ 1101100 \ 1101001 \ 1110011$_{2}$}\\
		{\footnotesize 52 \ 160 \ 157 \ 154 \ 151 \ 163$_{8}$} \ 
		{\footnotesize 42 \ 112 \ 111 \ 108 \ 105 \ 115$_{10}$} \ 
		{\footnotesize 2A \ 70 \ 6F \ 6C \ 69 \ 73$_{16}$}\\}
		\noindent\rule{\textwidth}{0.2pt}
		{\small find \$WWW -name}
		\begin{center}{\LARGE *polis:$\backslash$ #1}\end{center}
		Reihe:#2
		\noindent\rule{\textwidth}{0.2pt}
		{\centering{\footnotesize Autor: #3 \quad Version: #4 \quad Lizenz: #5}\\}
	\noindent\rule{\textwidth}{0.7pt}
}

\newcommand{\theme}[2]{
	\begin{center}{\Large\textbf{#1}}\end{center}
	\begin{center}{#2}\end{center}
}

\setkomafont{section}{\normalsize\selectfont\mdseries}
\setkomafont{subsection}{\normalsize\selectfont\mdseries}
\setkomafont{subsubsection}{\normalsize\selectfont\mdseries}



\begin{document}
% ========================================================
% INFORMATION
% ========================================================
%\header{topic}{series}{author}{version}{license}
\header{*p \& s:$\backslash$}{\$ Neue Zeiten!?}{Carolin Zöbelein\footnote{contact(at)carolin-zoebelein(dot)de, PGP: 0x927afd3cde47e13b}}{2015/01/10 \ - 0.0.0.0.de\_2FreundeEntdeckungsreise (old name: AaP-00-00-00-v0de)}{CC BY-ND 3.0 DE}

%\theme{titel}{subtitle}
\theme{Asterisk und Polis}{Zwei Freunde auf Entdeckungsreise}
% ========================================================
% CONTENT
% ========================================================
\textit{Asterisk} *, fälschlicherweise gerne mit dem allseits bekannten Gallier verwechselt, und sein Freund \textit{polis} (\textit{altgr.} Stadt oder Staat) finden sich, auf der Suche nach neuen Zivilisationen, in den unendlichen Weiten des Netzes, dort, wo noch nie ein Politiker (und Oma Frida) zuvor gewesen ist. 
\\
Auf ihren Reisen begegnen sie so manch seltsamen Gestalten. Dort, zwischen ewigen Hippies, Spielern, Alienentführten, Sozial-Medien-Zombies, Radikalen, Pädophilen, Drogenhändlern und den wahrhaft edlen Anbetern des fliegenden Spaghettimonsters, ja dort, gibt es viel Raum für Abenteuer und Verrücktes.
\\
Asterisk fühlt sich wohl. Hier ist sein zu Hause. Hier, wo sich Swingerclub und örtlicher Strickverein die selbe Adresse teilen. Auf manch seltsamen Wegen bewegt man sich hier. \\
Polis glaubt sich zunächst bei seinem wöchentlichen Bio-Einkauf zu befinden. Eine Welt, in der sich mit angebissenen Äpfeln, Himbeeren oder neuerdings auch Bananen fortbewegt werden kann. Auch so mancher anonymer Zwiebelknolle begegnet man. Er ist verschwirrt. \glqq Wo bleibt der Tierschutz, um sich um die vielen brennenden Füchse zu kümmern?\grqq, fragt er Asterisk. Der grinst und freut sich.
\\
Er als Grieche, aus Zeiten der großen Philosophen, ist erschüttert. Wo sind sie geblieben? Die politisch Interessierten, die Visionäre, die Querdenker, die Revolutionäre, die Kreativen und sonstige Exoten, die in der Vergangenheit die Gesellschaft nachhaltig geprägt haben. Wo ist die Gemeinschaft geblieben, in Zeiten des Sozialexhibitionismuskapitalismus? (Neue Worte erfinden konnte Polis schon immer gut. Das Handwerkszeug des Philosophen und Politikers in ihm.)
\\
Asterisk beruhigt ihn. \glqq Die gibt es noch! Vielfältiger als je zuvor. Und sie haben sich sogar weiter entwickelt! Neue Zeiten eröffnen noch nie zuvor dagewesene Möglichkeiten. Kunst, Musik, Kommunikation, Völkerverständigung, kultureller Austausch, Gemeinschaftsstrukturen und vor allem, erstmals echte grenzübergreifende, globale, Ideen und Bewegungen. Die Entstehung einer neuen Gesellschaft und einer neuen Form von dir, lieber Polis. Man kann dich überall und nirgends finden. Heute hier, morgen dort und doch stets präsent. Ich werde dich auf deinem Selbstfindungsprozess begleiten. Gemeinsam richten wir einen Blick auf die Vergangenheit, Gegenwart und wagen Zukunftsvisionen, die auf uns alle warten.\grqq
\\
\\
Lasst sie uns gemeinsam auf ihrer Entdeckungsreise begleiten!
% ========================================================
\end{document}
