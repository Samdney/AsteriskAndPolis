\documentclass[10pt, a4paper, oneside]{scrartcl}

\usepackage[T1]{fontenc}
\usepackage[utf8]{inputenc}
\usepackage[ngerman]{babel}

\usepackage{amsmath,marvosym}
\usepackage{amssymb}
\usepackage{graphicx}
\usepackage{float}
\usepackage{color}
\usepackage{xcolor}
\usepackage{url}
\usepackage{pdflscape}
\usepackage{multirow}
\usepackage{multicol}
\usepackage{tabularx}
\usepackage{longtable}
\usepackage{array}
\usepackage{listings}

\definecolor{dunkelgrau}{rgb}{0.8,0.8,0.8}
\definecolor{grau}{gray}{0.25}

\newcommand{\header}[5]{
	\noindent\rule{\textwidth}{0.7pt}
		{\centering{\footnotesize 0101010 \ 1110000 \ 1101111 \ 1101100 \ 1101001 \ 1110011$_{2}$}\\
		{\footnotesize 52 \ 160 \ 157 \ 154 \ 151 \ 163$_{8}$} \ 
		{\footnotesize 42 \ 112 \ 111 \ 108 \ 105 \ 115$_{10}$} \ 
		{\footnotesize 2A \ 70 \ 6F \ 6C \ 69 \ 73$_{16}$}\\}
		\noindent\rule{\textwidth}{0.2pt}
		{\small find \$WWW -name}
		\begin{center}{\LARGE *polis:$\backslash$ #1}\end{center}
		Reihe:#2
		\noindent\rule{\textwidth}{0.2pt}
		{\centering{\footnotesize Autor: #3 \quad Version: #4 \quad Lizenz: #5}\\}
	\noindent\rule{\textwidth}{0.7pt}
}

\newcommand{\theme}[2]{
	\begin{center}{\Large\textbf{#1}}\end{center}
	\begin{center}{#2}\end{center}
}

\setkomafont{section}{\normalsize\selectfont\mdseries}
\setkomafont{subsection}{\normalsize\selectfont\mdseries}
\setkomafont{subsubsection}{\normalsize\selectfont\mdseries}



\begin{document}
% ========================================================
% INFORMATION
% ========================================================
%\header{topic}{series}{author}{version}{license}
\header{*p \& s:$\backslash$}{\$ Neue Zeiten!?}{Carolin Zöbelein\footnote{contact(at)carolin-zoebelein(dot)de, PGP: 0x927afd3cde47e13b}}{2015/01/17 \ - 0.0.0.1.de\_LebenDatenreisenden (old name: AaP-00-00-01-v0de)}{CC BY-ND 3.0 DE}

%\theme{titel}{subtitle}
\theme{Asterisk: *p}{Das Leben eines Datenreisenden}
% ========================================================
% CONTENT
% ========================================================
Es war einmal ein britischer Schriftsteller namens \textit{Douglas Adams}. Der erschuf im Jahre 1978\footnote{\url{https://de.wikipedia.org/wiki/Per_Anhalter_durch_die_Galaxis}, 2015/01/14 - 04:31 PM CET} eine Hörspielserie mit dem Titel \textit{Per Anhalter durch die Galaxis}, gefolgt von einer fünfteiligen, gleichnamigen, Romanreihe in den Jahren 1979 - 1992. Nichts ahnend der Tragweite einer kleinen Entscheidung, die wohl für immer in die Geschichte eingehen wird. Die Antwort auf die Frage \textit{nach dem Leben, dem Universum und dem ganzen Rest} 
\begin{center}{\huge 42}\end{center}
Vielerlei Theorien ranken sich um diese Zahl und ihre wahre Bedeutung. Für die Einen der Ausdruck einer ganzen Kultur, für die Anderen Teufelswerk. Ja manch einer mag sie doch gar mit 666 gleichsetzen (obwohl das mathematisch nicht viel Sinn zu machen scheint!?). Doch welcher ominöse Zirkel verbirgt sich hinter der 42-Verschwörung? Dunkel gekleidete Gestalten, die das Tageslicht scheuen. Vielleicht Vampire auf der ständigen Suche nach dem golden farbigen Lebenssaft? Wer weiß das schon genau? Sprechen sie doch mit seltsamer Sprache unverständliche Dinge. Durchtrieben und hinterlistig, ja das müssen sie sein. Immer etwas Böses im Schilde führend.
\\
Auch \textit{Asterisk}, *, gehört dieser Spezies an. Bekannt als das \textit{Wildcard}\footnote{\textit{engl.} eine Spielkarte im Poker (Joker)/Platzhalter im Computerwesen} für eine beliebige Zeichenkette. \glqq Haben wir es doch gewusst, ihr sprecht mit fremden Zungen!\grqq, hört man es schallen. Allseits beliebt ist er, und nützlich. Einer der vieles kann. Steht er doch für so viele Dinge. Er akzeptiert alles, ohne Abweisung und Vorverurteilung. Er ist offen für jegliche Option und sei sie noch so Wage. Auch kritisch ist er. Denn manchmal gibt er auch nichts zurück. Dann, wenn er Dinge, die nicht den freien Lizenzbedingungen entsprechen, nach /dev/null verschoben hat.
\\
In Anbetracht dessen, mag man sich nun fragen, was sich einst der Erschaffer der ASCII-Tabelle, im Jahre 1963\footnote{\url{https://de.wikipedia.org/wiki/American_Standard_Code_for_Information_Interchange}, 2015/01/14 - 04:45 PM CET}, dachte. Denn suchen wir hier unseren Freund *, entdecken wir Erstaunliches. Wird er doch der Zahl $42_{10}$ zugeordnet. Und das, Jahre bevor zum ersten Mal Jemand per Anhalter durch die Galaxis flog. Seltsam. Der erste Schritt einer Sekte auf dem Weg zur Weltmacht? Durch Aliens manipulierte Gedankengänge? Oder doch der Teufel selbst?
\\
Asterisk ist somit viel mehr, als nur ein Zeichen auf der Tastatur, das mancher erst einmal für seine Steuererklärung suchen muss. Er ist das Sinnbild einer neuen Kultur mit unverwechselbaren Idealen, dessen Reise gerade erst begonnen hat.
% ========================================================
\end{document}
